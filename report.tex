\documentclass{article}

\usepackage[utf8x]{inputenc}%%permite acentos y otras bobadas
\usepackage[T1]{fontenc}%%hace que letras con acento sean una sola

\usepackage{indentfirst}%Indenta primera linea de sección

\usepackage{verbatimbox}%Caja verbatim configurable

\usepackage{enumitem}%Para añadir una nueva linea luego de cada item de una lista

\usepackage[a4paper, total={6in, 8in}]{geometry}%hoja A4 y sus medidas (texto: 6"x8")

\usepackage{changepage}%Para que la lista de items este indentada y se pueda separar en varias paginas

\usepackage{graphicx}%Para añadir imagenes
\usepackage{flafter}%Imagenes aparecen entre texto

\newcommand{\insertverbbox}{{\addvbuffer[6pt 3pt]{\theverbbox}\par}}

\title{\textbf{{\small Acerca del}\\
		Videojuego\\
		{\small usando el}\\
		Motor Phas3r\\
		{\small y el sistema de control de versiones}\\
		Git\\
		{\small y almacenado en}\\
		GitHub\footnote{https://github.com/EmmanuelMess/School-Phas3r-Game-Project}\\
		{\small trabajo práctico para}\\
		Programación Orientada A Objetos}\\
	Profesor: Mariano D'agostino}
\author{F Emmanuel Messulam\\
    \and Franco I Vallejos  Vigier\\
    \and \\\\
    Instituto Politécnico Superior de Rosario ``Gral. San Martin''}
\date{}

\begin{document}
    \begin{titlepage}
        \maketitle
        \thispagestyle{empty}
    \end{titlepage}
    \section*{Introducción}   
    [Descripción general de la solución construida]
    
	\section*{Desafios}
    [Principales desafíos enfrentados durante la construcción del proyecto.]
    	
   	\section*{Ejemplos}
   	
    \subsection*{Instancia de clase}
    
    \begin{verbbox}
    	
    \end{verbbox}
    \insertverbbox

    \subsection*{Método publico}
    
    \begin{verbbox}

    \end{verbbox}
    \insertverbbox
    
   	\subsection*{Método privado}
   	
   	\begin{verbbox}
   		
   	\end{verbbox}
   	\insertverbbox
    
   	\subsection*{Clase}
   	
   	\begin{verbbox}
   		
   	\end{verbbox}
   	\insertverbbox  
   	
   	\subsection*{Atributo de clase}
   	
   	\begin{verbbox}
   		
   	\end{verbbox}
   	\insertverbbox
   	
	\subsection*{Constructor}
	   	
	\begin{verbbox}
	   		
	\end{verbbox}
	\insertverbbox
    
   	\section*{Mejoras posibles}
   	
   	\subsection*{[Mejora 1]}
	
	[texto]
	
	\subsection*{[Mejora 2]}
	
	[texto]
	
	\subsection*{[Mejora 3]}

	[texto]

\end{document}
