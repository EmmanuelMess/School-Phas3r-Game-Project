\documentclass{article}

\usepackage[utf8x]{inputenc}%%permite acentos y otras bobadas
\usepackage[T1]{fontenc}%%hace que letras con acento sean una sola

\usepackage{indentfirst}%Indenta primera linea de sección

\usepackage{verbatimbox}%Caja verbatim configurable

\usepackage[a4paper, total={6in, 8in}]{geometry}%hoja A4 y sus medidas (texto: 6"x8")

\usepackage[hyphens]{url}%Para añadir links

\usepackage{csquotes}%%citas

\newcommand{\insertverbbox}{{\addvbuffer[6pt 3pt]{\theverbbox}\par}}

\urldef\inserturl\url{https://github.com/EmmanuelMess/School-Phas3r-Game-Project}

\title{\textbf{{\small Acerca del}\\
		Videojuego\\
		{\small usando el}\\
		Motor Phaser\\
		{\small y el sistema de control de versiones}\\
		Git\\
		{\small y almacenado en}\\
		GitHub\footnote{\inserturl}\\
		{\small trabajo práctico para}\\
		Programación Orientada A Objetos}\\
	Profesor: Mariano D'agostino}
\author{F Emmanuel Messulam\\
    \and Franco I Vallejos  Vigier\\
    \and \\\\
    Instituto Politécnico Superior de Rosario ``Gral. San Martin''}
\date{}

\begin{document}
    \begin{titlepage}
        \maketitle
        \thispagestyle{empty}
    \end{titlepage}
	\section*{Introducción}   
	%[Descripción general de la solución construida]
	Se presenta el trabajo practico, un juego donde el jugador tiene que luchar contra zombies para pasar el nivel. El nivel de dificultad es progresivo, es decir no se empieza con una oleada de 15. El nivel de dificultad crece exponencialmente, véase el array \verb|EnemigosACrear| en el archivo ``Principal.js''.
	
	A pesar de que el juego no posee tutorial, creemos intuitiva la forma de interaccion, (las teclas WASD para moverse, que son ampliamente usadas, y el espacio para disparar son una de las primeras cosas que, creemos, el jugador va a intentar).
	
	\section*{Desafios}
	%[Principales desafíos enfrentados durante la construcción del proyecto.]
	Desafíos no se nos presentaron en la forma en la que normalmente  aparecen en proyectos con cientos de miles, o millones, de lineas. Si bien hubo que lidiar con el desconocimiento general de algunas herramientas proveídas por la plataforma GitHub, como la ``Pull request'', que no existe en git. 
	
	No se presento mayor dilema a la hora de separar el proyecto en archivos, ya que no iba a ser mantenido por un largo periodo de tiempo, se decidió poner todo en un solo archivo ``Principal.js''.
	
	Tampoco se presento mayor dilema para separar todo en clases, ya que las ideas de ``objeto'' son muy fáciles de implementar en un videojuego, cosas como una `bala' o un `zombi' son claramente clases; y, claramente, el juego en sí es un objeto del cual solo existe una instancia, ``singleton''. Aunque esta ultima no esta expresada en código como tal, el juego es simplemente un diccionario de funciones, por la forma en que Phaser trata al juego y como ECMAScript trata a las clases y objetos.

   	\section*{Ejemplos}
   	
    \subsection*{Instancia de clase}
    
    \begin{verbbox}
    	
    \end{verbbox}
    \insertverbbox

    \subsection*{Método publico}
    
    \begin{verbbox}

    \end{verbbox}
    \insertverbbox
    
   	\subsection*{Método privado}
   	
   	\begin{verbbox}
   		
   	\end{verbbox}
   	\insertverbbox
    
   	\subsection*{Clase}
   	
   	\begin{verbbox}
   		
   	\end{verbbox}
   	\insertverbbox  
   	
   	\subsection*{Atributo de clase}
   	
   	\begin{verbbox}
   		
   	\end{verbbox}
   	\insertverbbox
   	
	\subsection*{Constructor}
	   	
	\begin{verbbox}
	   		
	\end{verbbox}
	\insertverbbox
    
   	\section*{Mejoras posibles}
   	\urldef\inserturl\url{https://github.com/EmmanuelMess/School-Phas3r-Game-Project/issues?q=is%3Aissue+is%3Aopen+sort%3Aupdated-desc+label%3Aenhancement}
   		
	Para una guía más actualizada y más completa véase la pagina de issues (filtrando por ``enhancements'') en GitHub\footnote{\inserturl}.
	
   	\subsection*{Un sistema de niveles}
	
	Más niveles, un menú con más niveles, y conforme se avance se desbloqueé nuevos personajes. Un sistema de progreso donde el jugador pueda avanzar y enfrentarse a enemigos más fuertes. Al implementar un sistema de logros, buscamos que el usuario vuelva a jugar niveles anteriores, por el simple hecho de adquirir contenido desbloqueable exclusivo (armas por ejemplo), el cual no sería posible de comprar o conseguir en la tienda.
	
	\subsection*{Un mecanismo de experiencia}
	
	Un sistema de experiencia con contenido desbloqueable para otorgar más fuerza en los ataques, creación de un HUD. Mejoras graduales al jugador ayudarian a contrarrestar los enemigos más fuertes si se implementaran más niveles. Implementar un BOSS al final de cada nivel.
	Un sistema que permita un juego cooperativo para el usuario, así como la posibilidad de escoger personajes, distintos, al comienzo de cada nuevo nivel. 
	Agregar dinámica al juego con consumibles, misiones secundarias, medallas o logros, además de vehículos. Nuestra idea en el desarrollo, fue realizar un juego en el cual se incremente rápidamente la dificultad del mismo, por eso sería ideal implementar un mecanismo de guardado o checkpoint.
	Implementar plataformas, con las cuales el usuario pueda interactuar y de ahí poder sacar ventaja de sus enemigos.
	
	\subsection*{Actualizar a la versión 3 del motor Phaser}
	
	Como dice el FAQ (preguntas frecuentes, por sus siglas en ingles) de Phaser 3 (traducción)\footnote{Véase la pregunta tres en \url{https://phaser.io/phaser3/faq}}:
	\begin{displayquote}
	 	...como v3 otorga un montón de nuevas formas de lograr objetivos comunes. Te darán la posibilidad de que tu código sea más claro y más estructurado. Aprovechalo.
	\end{displayquote}
	
	Pero aún hay pocos ejemplos y documentación con respecto a la nueva versión, por lo que se prefirió usar la v2. En un futuro debería moverse el proyecto para aprovechar la nueva versión.

\end{document}
